\documentclass{beamer}
\usepackage{hyperref}
\usetheme{Madrid}
\usepackage{listings}
\usepackage{xcolor}

% defining the syle for python code listing
\definecolor{codegreen}{rgb}{0,0.6,0}
\definecolor{codegray}{rgb}{0.5,0.5,0.5}
\definecolor{codepurple}{rgb}{0.58,0,0.82}
\definecolor{backcolour}{rgb}{0.95,0.95,0.92}
\lstdefinestyle{mystyle}{
    language=Python,
    backgroundcolor=\color{backcolour},   
    commentstyle=\color{codegreen},
    keywordstyle=\color{magenta},
    numberstyle=\tiny\color{codegray},
    stringstyle=\color{codepurple},
    basicstyle=\ttfamily\footnotesize,
    breakatwhitespace=false,         
    breaklines=true,                 
    captionpos=b,                    
    keepspaces=true,                 
    numbers=left,                    
    numbersep=5pt,                  
    showspaces=false,                
    showstringspaces=false,
    showtabs=false,                  
    tabsize=2
}
\lstset{style=mystyle}
%Information to be included in the title page:
\title{Data Strucutres and Algorithms}
\author{Nithin}
\institute{}
\date{\today}

\begin{document}

\frame{\titlepage}

\begin{frame}
    \frametitle{Table of Contents}
    \tableofcontents
\end{frame}
\section{Algorithm Analysis}
\subsection{Big-O}
\begin{frame}
\frametitle{What is Algorithm?}
As per \href{https://en.wikipedia.org/wiki/Donald_Knuth}{Donald Knuth}
\begin{alertblock}{Algorithm}
     A definite, effective and finite process that receives input and produces an output
\end{alertblock}
\begin{description}
    \item [Definite :] steps are clear, concise and unambigious
    \item [Effective :] you can perform each operation precisely 
    \item [Finite :] finite number of steps
\end{description}
\begin{alertblock}{Analysis}
    When two programs solve the same problem, Analysis is finding answer to the question which one is \alert{better}?
\end{alertblock}

\end{frame}


\begin{frame}{Better on?}
    \begin{itemize}
        \item <1-> Readability : \pause changes with programming language \pause
        \item <2-> Number of Lines : \pause changes with programming language \pause
        \item <3-> Amount of computing resources : \pause changes with programming language\pause
        \item <4-> Run time : \pause changes with processor speed, compiler and programminglanguage
    \end{itemize}
\end{frame}

\begin{frame}{An example : Checking the run time}
    \href{run://home/nithin/Work/python/dsa/big-o/}{our first example}
\end{frame}

\begin{frame}{Big-O Notation}
\begin{block}{Requirement}
    To charactrize an algorithm's efficiency in terms of execution time, independet of any particular program or computer
\end{block}
\begin{block}{Solution}
    To quanitfy the algorithm in terms of number of operations or steps 
\end{block}
\end{frame}

\begin{frame}
    \frametitle{$T(n)$}
    \begin{block}{}
        $T(n)$ is a function that indicates the time an algorithm takes to solve a problem of size $n$
    \end{block} 
\end{frame}

\begin{frame}[fragile]
\frametitle{Example 1}
    \begin{lstlisting}
        def sum_of_n(n):
            total = 0
            for i in range(n):
                total+=i
            return total
    
    \end{lstlisting}
\end{frame}

\begin{frame}
    \begin{itemize}
        \item For \textcolor{red}{sum\_of\_n}, we can take the basic compute step as the assignment operations
        \item In \textcolor{red}{sum\_of\_n} following are the assignment operations 
        \begin{itemize}
            \item \textcolor{red}{sum = 0}
            \item \textcolor{red}{sum$+=n$}
        \end{itemize}
        \item $T(n) = n+1 $
        \item We are only interested in the dominant term in $T(n)$, beacuse as $n$ increases faster compared to other terms, i.e it overpowers the rest
    \end{itemize}
    \begin{alertblock}{Big-O}
        The dominant term in $T(n)$ , which can be termed as \alert{order of magnitude} function. Big-O $\implies$ Biggest Order
        
    \end{alertblock}

\end{frame}

\begin{frame}
frametitle{Common Big-O functions}

\end{frame}
\begin{frame}[fragile]
    \frametitle{Quiz 1}
    What is the Big-O for the program given below :
    \begin{lstlisting}
        a=5
        b=6
        c = 10
        for i in range(n):
            for j in range(n): 
                x=i*i 
                y=j*j
                z=i*j
        for k in range(n): 
            w = a * k + 45
            v=b*b 
        d = 33
    \end{lstlisting}
    \pause 
    $T(n)  = 3n^{2} + 2n + 4 \implies O(n^{2})$
    
    
\end{frame}
\subsection{Anagram Example}
\begin{frame}
    \frametitle{Anagram}
    \begin{block}
        A string is an anagram of another if second is simply a rearrangment of the first. For example \alert{python} and \alert{typhon}
    \end{block}
    
\end{frame}

\begin{frame}[fragile]
    \frametitle{Solution 1: Checking off }
    \begin{lstlisting}{language=Python}
        def anagram_sol1(word1, word2):
            word2_list = list(word2)
            index1=0
            is_anagram = True
            while index1 < len(word1) and is_anagram:
                index2=0
                is_continue = True
                while index2 < len(word2_list) and is_continue:
                    if word1[index1] == word2_list[index2]:
                        word2_list[index2] = None 
                        is_anagram = True
                        is_continue = False
                    else:
                        is_anagram = False
                        is_continue = True
                        index2+=1
                index1+=1
            return is_anagram       
    \end{lstlisting}
\end{frame}
\begin{frame}{Solution 1: Big-O}
    \begin{example}{}
        Each letter in word1 has to iterate a maximum of n locations to find a match. That is 
        \[ \color{violet} \sum_{i=1}^{n} i  = \frac{n(n+1)}{2} = \frac{n^{2} + n}{2} \]
        
    \end{example}
    
\end{frame}
\begin{frame}[fragile]{Solution 2: Sort and Compare}
    \begin{lstlisting}{language=Python}
        def anagram_sol2(word1, word2):
            word1_list = list(word1)
            word2_list = list(word2)
            word1_list.sort()
            word2_list.sort()
            index=0
            while index < len(word1_list):
                if word1_list[index] != word2_list[index]:
                    return False
                index+=1
            return True
    \end{lstlisting}
    
\end{frame}
\begin{frame}{Solution 2: Big-O}
    \begin{block}{}
        There is a single iteration of n if there is a match but the sort operation takes the precedence due to it's $\color{violet}O(n^{2})$ or $\color{violet}O(n\log n)$ complexity          
    \end{block}
    
\end{frame}

\begin{frame}{Solution 3: Brute Force}
    \begin{block}{}
        This tries to exhaust all possibilities. Here we genrate all possible anagrams of \alert{word1} and matches this with \alert{word2}. For a word of length $\color{violet} n$, there are $\color{violet}n!$
    \end{block}
    
\end{frame}

\begin{frame}[fragile]{Solution 4: Count and Compare}
    \begin{lstlisting}
    def anagram_sol4(word1,word2):
        counter1 = [0]*26
        counter2 = [0]*26
        offset = ord('a')
        for letter in word1:
            counter1[ord(letter)-offset]+=1
        for letter in word2:
            counter2[ord(letter)-offset]+=1
        for index in range(26):
            if counter1[index] != counter2[index]:
                return False
        return True
    \end{lstlisting}
    \pause
    \begin{block}{}
        \[\color{violet} T(n) = 2n + 26 \implies O(n) \]
        The solution above can run in linear time but required more space requirements than the other solutions
    \end{block}
    
\end{frame}

\section{Basic Data Strucutres}
\subsection{Linear Data Strucutre}
\begin{frame}{Linear Strucutres}
    \begin{alertblock}{What is Linear Data Strucutre?}
        \begin{itemize}
            \item Data structures in which each element stays its position relative to the elements before and after.
            \item Examples are Stacks, Queues, Deques and Lists
            \item The difference are in the way items are added or removed
        \end{itemize}
    \end{alertblock}
\end{frame}
\subsection{Stack}
\begin{frame}{Stack}
    \begin{alertblock}{What is a Stack?}
        A \alert{stack} is an ordered collection of items where the addition of new item and the removal of existing items always takes place at the same end 
    \end{alertblock}
    \begin{itemize}
        \item also known as \alert{Last-In-First-Out(LIFO)}
        \item Newer items are in the top and older items are at the bottom
        \item Browser \alert{Back} button is an example of stack
    \end{itemize}
\end{frame}

\begin{frame}{Stack-ADT: Python Implementation}
    \begin{itemize}
        \item \textcolor{red}{\textbf{Stack()}} : creates new stack. Needs no parameters rather returns an empty stack
        \item  \textcolor{red}{\textbf{push(item)}} :adds a new item to top of the stack. It needs the item and returns nothing
        \item \textcolor{red}{\textbf{pop()}} : removes the top item from the stack. It needs no parameter and return the item. stack is modified
        \item \textcolor{red}{\textbf{peek()}} : returns top item from the stack. No item parameter is required and did not modify the stack
        \item \textcolor{red}{\textbf{is\_empty()}} : tests to see whether stack is empty. It needs no parameter and returns a boolean value
        \item \textcolor{red}{\textbf{size()}} :  returns the number of items in the stack. It needs no parameter and returns an integer value
    \end{itemize}
\end{frame}
\begin{frame}[fragile]{Stack ADT: Python Implementation}
    \begin{lstlisting}
        class Stack:
            def __init__(self) -> None:
                self.items = []

            def is_empty(self):
                return self.items == []

            def push(self, item):
                return self.items.append(item)

            def pop(self):
                return self.items.pop()

            def peek(self):
                return self.items[-1]

            def size(self):
                return len(self.items)
    \end{lstlisting}
\end{frame}
\begin{frame}[fragile]
    \frametitle{Stack Example: Paranthesis Checker}
    \begin{lstlisting}
        def para_check(symbol_str):
            # creating symbol templates and the key - value pairs
            open_symbol = {'(':0, '[':1, '{':2}
            close_symbol = {')':0, ']':1, '}':2}
            # initialising the stack
            para_stack = Stack()
            #to check the unbalance condition during symbol iteration and exit the loop is exists 
            is_check = True
            index = 0
            while index < len(symbol_str) and is_check:
                symbol = symbol_str[index]
                #checking for opening symbols
                if symbol in open_symbol.keys():
                    para_stack.push(symbol)
    \end{lstlisting}
\end{frame}

\begin{frame}[fragile]    
    \begin{lstlisting}
            #checking for closing symbols
            elif symbol in close_symbol.keys():
                # check is stack is empty and still a close symbol exists resulting in unbalance
                if para_stack.is_empty():
                    # exit the loop if exists
                    is_check = False
                # check if the closing symbol is equal to opening symbol , if not, exit the loop
                elif close_symbol[symbol] != open_symbol[para_stack.peek()] :
                    is_check = False 
                else:
                    # if both of the above condition fails, then there is a balance exists and pop out the symbol
                    para_stack.pop()
    \end{lstlisting}
\end{frame}
    
\begin{frame}[fragile]
    \begin{lstlisting}
                else:
                    pass
                index+=1
            # to check if the stack is empty and check the balance flag 
            if para_stack.is_empty() and is_check:
                return f"balanced"
            else:
                return f"unbalanced"

    \end{lstlisting}
\end{frame}
    
\end{document}




















































