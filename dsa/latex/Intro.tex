\documentclass{beamer}
\usepackage{hyperref}
\usetheme{Madrid}
%Information to be included in the title page:
\title{Data Strucutres and Algorithms}
\author{Nithin}
\institute{}
\date{\today}

\begin{document}

\frame{\titlepage}

\begin{frame}
    \frametitle{Table of Contents}
    \tableofcontents
\end{frame}
\section{Algorithm Analysis}
\begin{frame}
\frametitle{What is Algorithm?}
As per \href{https://en.wikipedia.org/wiki/Donald_Knuth}{Donald Knuth}
\begin{alertblock}{Algorithm}
     A definite, effective and finite process that receives input and produces an output
\end{alertblock}
\begin{description}
    \item [Definite :] steps are clear, concise and unambigious
    \item [Effective :] you can perform each operation precisely 
    \item [Finite :] finite number of steps
\end{description}
\begin{alertblock}{Analysis}
    When two programs solve the same problem, Analysis is finding answer to the question which one is \alert{better}?
\end{alertblock}

\end{frame}


\begin{frame}{Better on ?}
    \begin{itemize}
        \item <1-> Readability : \pause changes with programming language \pause
        \item <2-> Number of Lines : \pause changes with programming language \pause
        \item <3-> Amount of computing resources : \pause changes with programming language\pause
        \item <4-> Running time : \pause changes with processor speed and language \pause
    \end{itemize}
\end{frame}




\end{document}