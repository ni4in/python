\documentclass{beamer}
\usepackage{hyperref}
\usetheme{Madrid}
\usepackage{listings}
\usepackage{xcolor}

% defining the syle for python code listing
\definecolor{codegreen}{rgb}{0,0.6,0}
\definecolor{codegray}{rgb}{0.5,0.5,0.5}
\definecolor{codepurple}{rgb}{0.58,0,0.82}
\definecolor{backcolour}{rgb}{0.95,0.95,0.92}
\lstdefinestyle{mystyle}{
    language=Python,
    backgroundcolor=\color{backcolour},   
    commentstyle=\color{codegreen},
    keywordstyle=\color{magenta},
    numberstyle=\tiny\color{codegray},
    stringstyle=\color{codepurple},
    basicstyle=\ttfamily\footnotesize,
    breakatwhitespace=false,         
    breaklines=true,                 
    captionpos=b,                    
    keepspaces=true,                 
    numbers=left,                    
    numbersep=5pt,                  
    showspaces=false,                
    showstringspaces=false,
    showtabs=false,                  
    tabsize=2
}
\lstset{style=mystyle}
%Information to be included in the title page:
\title{Data Strucutres and Algorithms}
\author{Nithin}
\institute{}
\date{\today}

\begin{document}

\frame{\titlepage}

\begin{frame}
    \frametitle{Table of Contents}
    \tableofcontents
\end{frame}
\section{Algorithm Analysis}
\begin{frame}
\frametitle{What is Algorithm?}
As per \href{https://en.wikipedia.org/wiki/Donald_Knuth}{Donald Knuth}
\begin{alertblock}{Algorithm}
     A definite, effective and finite process that receives input and produces an output
\end{alertblock}
\begin{description}
    \item [Definite :] steps are clear, concise and unambigious
    \item [Effective :] you can perform each operation precisely 
    \item [Finite :] finite number of steps
\end{description}
\begin{alertblock}{Analysis}
    When two programs solve the same problem, Analysis is finding answer to the question which one is \alert{better}?
\end{alertblock}

\end{frame}


\begin{frame}{Better on?}
    \begin{itemize}
        \item <1-> Readability : \pause changes with programming language \pause
        \item <2-> Number of Lines : \pause changes with programming language \pause
        \item <3-> Amount of computing resources : \pause changes with programming language\pause
        \item <4-> Run time : \pause changes with processor speed, compiler and programminglanguage \pause
    \end{itemize}
\end{frame}

\begin{frame}{An example : Checking the run time}
    \href{run://home/nithin/Work/python/dsa/big-o/}{our first example}
\end{frame}

\begin{frame}{Big-O Notation}
\begin{block}{Requirement}
    To charactrize an algorithm's efficiency in terms of execution time, independet of any particular program or computer
\end{block}
\begin{block}{Solution}
    To quanitfy the algorithm in terms of number of operations or steps 
\end{block}
\end{frame}

\begin{frame}
    \frametitle{$T(n)$}
    \begin{block}{}
        $T(n)$ is a function that indicates the time an algorithm takes to solve a problem of size $n$
    \end{block} 
\end{frame}

\begin{frame}[fragile]
\frametitle{Example 1}
    \begin{lstlisting}
        def sum_of_n(n):
            total = 0
            for i in range(n):
                total+=i
            return total
    
    \end{lstlisting}
\end{frame}

\begin{frame}
    \begin{itemize}
        \item For \textcolor{red}{sum\_of\_n}, we can take the basic compute step as the assignment operations
        \item In \textcolor{red}{sum\_of\_n} following are the assignment operations 
        \begin{itemize}
            \item \textcolor{red}{sum = 0}
            \item \textcolor{red}{sum$+=n$}
        \end{itemize}
        \item $T(n) = n+1 $
        \item We are only interested in the dominant term in $T(n)$, beacuse as $n$ increases faster compared to other terms, i.e it overpowers the rest
    \end{itemize}
    \begin{alertblock}{Big-O}
        The dominant term in $T(n)$ , which can be termed as \alert{order of magnitude} function. Big-O $\implies$ Biggest Order
        
    \end{alertblock}

\end{frame}

\end{document}