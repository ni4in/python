\documentclass{beamer}
\usepackage{hyperref}
\usetheme{Madrid}
\usepackage{listings}
\usepackage{xcolor}

% defining the syle for python code listing
\definecolor{codegreen}{rgb}{0,0.6,0}
\definecolor{codegray}{rgb}{0.5,0.5,0.5}
\definecolor{codepurple}{rgb}{0.58,0,0.82}
\definecolor{backcolour}{rgb}{0.95,0.95,0.92}
\lstdefinestyle{mystyle}{
    language=Python,
    backgroundcolor=\color{backcolour},   
    commentstyle=\color{codegreen},
    keywordstyle=\color{magenta},
    numberstyle=\tiny\color{codegray},
    stringstyle=\color{codepurple},
    basicstyle=\ttfamily\footnotesize,
    breakatwhitespace=false,         
    breaklines=true,                 
    captionpos=b,                    
    keepspaces=true,                 
    numbers=left,                    
    numbersep=5pt,                  
    showspaces=false,                
    showstringspaces=false,
    showtabs=false,                  
    tabsize=2
}
\lstset{style=mystyle}
%Information to be included in the title page:
\title{Data Strucutres and Algorithms}
\author{Nithin}
\institute{}
\date{\today}

\begin{document}

\frame{\titlepage}

\begin{frame}
    \frametitle{Table of Contents}
    \tableofcontents
\end{frame}

\begin{frame}{Stack-ADT: Python Implementation}
    \begin{itemize}
        \item \textcolor{red}{\textbf{Stack()}} : creates new stack. Needs no parameters rather returns an empty stack
        \item  \textcolor{red}{\textbf{push(item)}} :adds a new item to top of the stack. It needs the item and returns nothing
        \item \textcolor{red}{\textbf{pop()}} : removes the top item from the stack. It needs no parameter and return the item. stack is modified
        \item \textcolor{red}{\textbf{peek()}} : returns top item from the stack. No item parameter is required and did not modify the stack
        \item \textcolor{red}{\textbf{is\_empty()}} : tests to see whether stack is empty. It needs no parameter and returns a boolean value
        \item \textcolor{red}{\textbf{size()}} :  returns the number of items in the stack. It needs no parameter and returns an integer value
    \end{itemize}
\end{frame}
\begin{frame}[fragile]{Stack ADT: Python Implementation}
    \begin{lstlisting}
        class Stack:
            def __init__(self) -> None:
                self.items = []

            def is_empty(self):
                return self.items == []

            def push(self, item):
                return self.items.append(item)

            def pop(self):
                return self.items.pop()

            def peek(self):
                return self.items[-1]

            def size(self):
                return len(self.items)
    \end{lstlisting}
\end{frame}

\end{document}